
\chapter{April 2018}
\label{ch:april-2018}

\section{ROS}
\paragraph{Janne Spijkervet}
While starting to program a new framework for the development of algorithms and programs for the Pepper robot, I stumbled upon ROS (Robot Operating System). Much of the design I wanted to make for my framework (publisher/subscriber programs, nodes) were already tested and tried in the ROS framework. I decided to port the work I had done to ROS (face recognition, text-to-speech, speech-to-text).

\section{ROS2}
\paragraph{Janne Spijkervet}
While looking through ROS' documentation, I was more tended to choose ROS2 for our framework. But, as its documentation is very scarce compared to ROS, I figured it would be better to go with ROS for now until I am familiar with the framework. ROS2 is preffered for us, because we often work with low-quality, unstable (wireless) networks. Also, we prefer real-time processing over anything else. ROS2 is built on newer protocols (DDS). \textbf{In the future, moving forward to ROS2 is very recommended!}


\section{Repositories}\label{sec:repositories}
\paragraph{Janne Spijkervet}
A new repository group was made that holds all ROS packages developed for the Intelligent Robotics Lab. This was done to make implementations of algorithms and other code more re-usable for future work. It is important to implement standards for robotics programming. ROS is one of such platforms that envisions this for future research and practical implementations of robotics software. The repositories can be found at: \url{https://gitlab.com/uva-robotics}.

\section{Workstations}\label{sec:workstations}
\paragraph{Janne Spijkervet}
\newthought{ROS} was installed on both workstations at the lab. This was done to speed up the work that involves the use of a GPU, since most students only have a laptop to work on.
